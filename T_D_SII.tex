\documentclass[12pt,letterpaper]{report}
\usepackage[utf8]{inputenc}
\usepackage[spanish]{babel}
\usepackage{amsmath}
\usepackage{amsfonts}
\usepackage{amssymb}
\usepackage{graphicx}
\usepackage[left=3cm,right=2.5cm,top=3cm,bottom=2cm]{geometry}
\author{Fernando Manzanares}
\title{Seminario II}
\setlength{\parskip}{4mm}
\usepackage{ragged2e}
\justifying
\usepackage{longtable}
\usepackage{Sweave}
\setlength\parindent{1.25cm}
\usepackage{setspace}
\usepackage{titlesec} 
\usepackage{xcolor}
\titleformat{\chapter}[display]{\bfseries\Huge}{Capítulo \ \Huge\thechapter.}{0.5em}{}
\titlespacing{\chapter}{0pt}{0ex}{1pc}
\usepackage{upgreek}
\widowpenalty=10000  
\clubpenalty=10000
\usepackage{multirow}
\setcounter{secnumdepth}{3} 
\setcounter{tocdepth}{4} 
\usepackage{fancyhdr}
\renewcommand{\labelitemi}{•}
\usepackage{apalike}
\usepackage{cite}

\begin{document}
\Sconcordance{concordance:T_D_SII.tex:T_D_SII.Rnw:%
1 21 1 1 0 2 1}
\Sconcordance{concordance:T_D_SII.tex:./Portada.Rnw:ofs 25:%
1 28 1}
\Sconcordance{concordance:T_D_SII.tex:./Indice.Rnw:ofs 54:%
1}
\Sconcordance{concordance:T_D_SII.tex:./Abstract.Rnw:ofs 55:%
1 2 1}
\Sconcordance{concordance:T_D_SII.tex:./Introduccion.Rnw:ofs 58:%
1 20 1}
\Sconcordance{concordance:T_D_SII.tex:T_D_SII.Rnw:ofs 79:%
29 6 1}


\spacing{1.5}

\thispagestyle{empty}

\begin{center}
UNIVERSIDAD DE EL SALVADOR \\ FACULTAD MULTIDISCIPLINARIA DE OCCIDENTE \\ DEPARTAMENTO DE MATEMÁTICAS \\
LICENCIATURA EN ESTADÍSTICA\\
\bigskip
\includegraphics [width=5cm,height=6cm]{Minerva}
\bigskip
\\ ANALISIS DE LA EFICIENCIA ENERGETICA EN LA UNIVERSIDAD DE EL SALVADOR FACULTAD MULTIDISCIPLINARIA DE OCCIDENTE \\
\bigskip
\textbf{MATERIA:}\\ 
SEMINARIO II\\
\textbf{ALUMNO:}\\ 
FERNANDO ERNESTO MANZANARES MORÁN\\
\bigskip
\bigskip
\bigskip
\bigskip
\bigskip
\bigskip

\textbf{SANTA ANA,    DICIEMBRE 2015,   EL SALVADOR}\\
\textbf{CENTRO AMÉRICA}\\


\end{center}


\spacing{1.0}

\tableofcontents 
\thispagestyle{empty}
\spacing{1.5}
\chapter*{Abstract}
\addcontentsline{toc}{chapter}{Abstract}
\pagenumbering{arabic}
Erase una vez...
\chapter*{Introducción}
\addcontentsline{toc}{chapter}{Introducción}

En la actualidad el medio ambiente es uno de los factores más tomados en cuenta con respecto
a la eficiente utilización de los recursos que éste entrega, además de la utilización de técnicas
estadísticas que aporten información para la correcta interpretación de la eficiencia con la que
dichos recursos son utilizados, estas son unas de las razones por la que investigaciones como esta
son necesarios para alcanzar la contribución efectiva en la protección del ambiente, en
términos generales, esta investigacion tiene como objetivo la provisión de 
un diseño favorable en cuanto analsis estadistico, sustentabilidad
ambiental, para alcanzar la eficiencia energética,
normalmente, los aspectos energéticos en cuanto a eficiencia están asociados a la especificación de
las potencias o capacidades de los equipos utilizados en la localización física y geográfica
donde se se encuentran ubicados, y su objetivo principal es alcanzar la utilización máxima de la
capacidad de los aparatos eléctricos con el menor consumo de energía posible, es decir a
través de actividades y tareas como el mantenimiento de sistemas y aparatos eléctricos, por esa
razón se busca la creación de una propuesta para alcanzar dicha eficiencia con el
fin de crear conciencia ambiental en las personas en general, en la UES FMOcc y lograr la
contribución al ambiente a través del ahorro de energía, además de sentar un precedente para
el análisis de datos, que se realizara a través de una serie de tiempo con el objetivo de alcanzar
el preciso análisis de los datos y así tomar la decisiones que generen mayor impacto ambiental.
\chapter{El Problema}
\section{Planteamiento del problema}
Desde hace ya varios años, la protección al medio ambiente se ha ido convirtiendo en un tema de mucha importancia,
tanto así, que varios países del mundo han comenzado a adoptar políticas para la conservación de este, dichas politicas van desde, el tratamiento desechos solidos hasta el ahorro de energía y cada día que pasa se adoptan muchas mas; entonces la NO protección del ambiente esta generando dichas reacciones en las poblaciones de todos los países alrededor del mundo, ya que en la actualidad se viven problemas como la contaminación y escasez de agua, contaminación del aire, degradación de los suelos y descenso de la productividad agrícola, deforestación, residuos sólidos y peligrosos, pérdida de la diversidad biológica, desgaste de la capa de ozono y cambios climáticos,todos estos problemas afectan a los países, de manera desigual, según el grado de su desarrollo, de su estructura económica y de las políticas ambientales que aplican, es decir para combatir el deterioro ambiental se llevan a cabo dos tipos de políticas; las que procuran relacionar el desarrollo con el medio ambiente a nivel general de población, recursos, legislación y tecnologías, y las que se orientan a problemas específicos, uno de los factores mas importantes al interior de estas problemáticas es el ahorro de energía, este factor entra en el segundo tipo de política anteriormente mencionada, es por eso que paises desarrollados están comenzando a implementar medidas que buscan dicho ahorro y ademas buscan consumir energía pero de manera eficiente.

En El Salvador, el problema de la eficiencia enérgetica no ha sido abordado desde el punto de vista estadistico, es decir, la aplicación de tecnicas estadisticas para investigar dicho problema,
normalmente, los aspectos energéticos en cuanto a eficiencia están asociados a la especificación de
las potencias o capacidades de los equipos utilizados en la localización física y geográfica
donde se se encuentran ubicados, y su objetivo principal es alcanzar la utilización máxima de la
capacidad de los aparatos eléctricos con el menor consumo de energía posible, por esa razon esta investigación busca la implementación de una serie temporal para el analisis de los datos del consumo electrico mensual en la Universidad de El Salvador Facultad Multidisciplinaria de Occidente (UES FMOcc), además, desarrollar proyeciones apartir de dicho modelo para analizar e interpretar el comportamiento del consumo eléctrico y sentar un presedente a nivel academico y teórico sobre el abordaje que se le pueden dar a este tipo de problemas, a estose agrega, contribuir en la protección del ambiente y generar conciencia para cambiar los habitos del consumo electrico, en la misma linea tambien pretende responder a las siguientes preguntas ¿Será eficiente el consumo eléctrico en la UES FMOcc? ¿Es posible alcanzar la eficiencia enérgetica en los edificios de la UES FMOcc? ¿Está el consumo energetico en la UES FMOcc creciendo a través del tiempo? ¿Cuáles son los periodos de tiempo donde se consume mas energia eléctrica? ¿Cuánta energía podria ahorrase si se lograra la eficiencia enérgetica?

\newpage
\section{Justificación}
Esta investigación pretende analizar y desarrollar proyeciones sobre los datos del consumo electrico en la UES FMOcc, para descubrir si existe o no eficiencia energetica además de la aplicación de un modelo estadistico autoregresivo integrado de media movil para datos estacionales y de esta manera poder interpretar los datos del consumo energetico de manera nueva y mas precisa, la importancia de realizar esta investigación se basa principlamente en el aporte teórico nuevo que sentara una base para el desarrollo de futuras investigaciones de esta índole, además de la aplicacion de tecnicas propiamente estadisticas para el monitoreo del comportamiento del consumo enérgetico en la UES FMOcc, esto con el fin de identificar los espacios de de tiempo donde se consume la mayor cantidad de energía eléctrica y de esta manera poder abordar de manera efectiva la problemática de eficiencia enérgetica, a esto se agrega que esta investigación tracenderá en el tiempo y será util durante los proximos 18 meses apartir de agosto de 2015 debido al tipo de tecnica utilizada para desarrollar las proyeciones y por último la contribución a la protección del ambiente una vez alcanzada la eficiencia en los sistemas electricos de la UES FMOcc, en el aspecto económico la realización de esta investigación resultaria favorable ya que la eficiencia enérgetica ademas de contribuir al ambiente sugiere una reducción en los costos del consumo enérgetico. 


\newpage
\section{Objetivos de la investigación}
\subsection{Objetivo general}

\begin{itemize}
\item Analizar los datos del consumo eléctrico a través de la utilización de una serie
temporal lineal o no lineal, para la interpretación de el comportamiento del consumo enérgetico a través del tiempo y si existe eficiencia energética de la UES FMOcc en los próximos 18 meses.
\end{itemize}

\subsection{Objetivos específicos}

\begin{itemize}
\item Desarrollar y analizar las proyecciones de los datos del consumo eléctrico de la
FMOcc a través de modelos estadísticos de series temporales para los próximos 18
meses.

\item Identificar el periodo de tiempo donde el consumo energetico es mas bajo y analogamente donde es mas alto.

\item Conecer la cantidad de energia y costos que podrian reducirse para contribuir al ambiente si se lograra la eficiencia energetica.  

\end{itemize}

\newpage
\section{Hipótesis}
\textbf{ Hipotesis 1.} 

$H_0$ No existe eficiencia energetica en los edificios de la Universidad de El Salvador Facultad Multidisciplinaria de Occidente.

$H_1$ Existe eficiencia energetica en los edificios de la Universidad de El Salvador Facultad Multidisciplinaria de Occidente.

\textbf{ Hipotesis 2.}

$H_0$ El consumo energetico en la Universidad de El Salvador Facultad Multidisciplinaria de Occidente es creciente a través del tiempo.

$H_1$ El consumo energetico en la Universidad de El Salvador Facultad Multidisciplinaria de Occidente no es creciente a través del tiempo.


\textbf{ Hipotesis 3.}

$H_0$ Los periodos donde más energia electrica se consume son durante los ciclos academicos desarrollados en la Universidad de El Salvador Facultad Multidisciplinaria de Occidente. 

$H_1$ Los periodos donde más energia electrica se consume no son durante los ciclos academicos desarrollados en la Universidad de El Salvador Facultad Multidisciplinaria de Occidente.





\chapter{Fundamentación teórica}
\section{Antecedentes}
\section{Teorías acerca de la eficiencia enérgetica}
\section{Teorías estadísticas}

\nocite{*}
\bibliographystyle{apalike}
\bibliography{SIIB}

\end{document}





